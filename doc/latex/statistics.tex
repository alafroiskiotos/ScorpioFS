\documentclass[a4paper]{article}
\usepackage{xgreek}
\usepackage{xltxtra}
\usepackage{graphics}
\setlength{\topmargin}{0in}
\setlength{\oddsidemargin}{0in}
\setlength{\evensidemargin}{0in}
\setlength{\textheight}{9in}
\setlength{\textwidth}{6.25in}
\setmainfont[Mapping=tex-text]{GFS Didot}

\begin{document}
\section{Πειράματα}
Για την κατανόηση της συμπεριφοράς της εφαρμογής και του Chord δικτύου,
διεξάγουμε κάποια χαρακτηριστικά πειράματα και εξάγουμε διάφορα στατιστικά. Τα
πειράματα έγιναν σε εργαστηριακό περιβάλλον με 50 (more or less) μηχανήματα.

Αρχικά μας ενδιαφέρει ο χρόνος αποθήκευσης αρχείων στο \emph{ScorpioFS}. Για το
λόγο αυτό διεξάγουμε πειράματα αποθήκευσης αρχείων διαφόρων μεγεθών σε ένα φάσμα
από Χ MB έως Y MB. Το \emph{ScorpioFS} χωρίζει τα αρχεία σε chunks μεγέθους 1
MB. Επομένως ειδική περίπτωση που μας ενδιαφέρει να μελετήσουμε είναι η
αποθήκευση αρχείων μεγέθους μικρότερου του ενός MB.

Ένα μέγεθος του συστήματος που επηρεάζει αισθητά τη συμπεριφορά του δικτύου
είναι το replication factor. Δηλαδή πόσες φορές θα αναπαράγει ένα chunk ώστε να
το αποθηκεύσει στους κόμβους του δικτύου. Θέτοντας διάφορες τιμές στη μεταβλητή
αυτή αυξάνεται ή μειώνεται η κίνηση του δικτύου.

Ο χρήστης που χρησιμοποιεί το \emph{ScorpioFS} αποθηκεύει αρχεία στο δίκτυο με
τελικό σκοπό κάποια στιγμή να τα ανακτήσει. Ως εκ τούτου μας ενδιαφέρει η
συμπεριφορά του συστήματος κατά την ανάκτηση διάφορων αρχείων. Έτσι διεξάγουμε
πειράματα για την ανάκτηση αρχείων τα οποία βρίσκονται είτε σε απομακρυσμένους
κόμβους είτε στην τοπική αποθήκη του κόμβου. Και εδώ υπάρχει η ειδική περίπτωση
να μην χρειαστεί να ανακτήσουμε όλα τα chunks ενός αρχείου. Για παράδειγμα στην
επισκόπηση μιας ταινίας, που δεν βλέπουμε όλα τα frames, τότε γίνεται ανάκτηση
μόνο των chunks που περιέχουν τα frames που βλέπουμε.

Το \emph{ScorpioFS} είναι μία δικτυακή εφαρμογή. Τα αρχεία αποθηκεύονται σε
διάφορους κόμβους του δικτύου. Επίσης το δίκτυο δεν είναι σταθερό. Νέοι κόμβοι
μπαίνουν στο δίκτυο και υπάρχοντες κόμβοι αποχωρούν. Αυτό έχει άμεση συνέπεια
στο σύστημα. Όσοι κόμβοι μπουν στο δίκτυο θα πρέπει να ενημερωθούν για τα chunks
που είναι υπεύθυνοι. Από την άλλη μεριά, όσοι κόμβοι φύγουν από το δίκτυο θα
πρέπει να ενημερώσουν τους υπόλοιπους. Αυτό που ενδιαφέρει το χρήστη είναι να
μπορεί σε κάθε στιγμή να ανακτήσει τα αρχεία του χωρίς να τον ενδιαφέρει το
δίκτυο των κόμβων. Επομένως διεξάγουμε πειράματα με δυναμική συμπεριφορά του
δικτύου, με κόμβους να μπαίνουν στο δίκτυο και άλλους να αποχωρούν, ώστε να
προσεγγίσουμε ένα ρεαλιστικό μοντέλο.

Για να μπορέσουμε να καταγράψουμε τη συμπεριφορά της εφαρμογής στις παραπάνω
συνθήκες θα πρέπει κάθε φορά να κρατάμε κάποια χαρακτηριστικά. Ενδεικτικά,
ορισμένα από αυτά είναι το μέγεθος της αποθήκης του κάθε κόμβου σε μια
συγκεκριμένη χρονική στιγμή, πόσο χρόνο συμμετέχει ένας κόμβος στο δίκτυο
(uptime), τον αριθμό των chunks που έχει ένας κόμβος στην αποθήκη του, τον
αριθμό των αιτήσεων για ανάκτηση των πιο ``δημοφιλών'' chunks καθώς και το
συνολικό αριθμό αιτήσεων για ανάκτηση και αποθήκευση κάθε κόμβου.
\end{document}
